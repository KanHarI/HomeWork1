\documentclass[12pt]{article}
 \usepackage[margin=1in]{geometry} 
\usepackage{amsmath,amsthm,amssymb,amsfonts}
 
\newcommand{\N}{\mathbb{N}}
\newcommand{\Z}{\mathbb{Z}}
 
\newenvironment{problem}[2][Problem]{\begin{trivlist}
\item[\hskip \labelsep {\bfseries #1}\hskip \labelsep {\bfseries #2.}]}{\end{trivlist}}
%If you want to title your bold things something different just make another thing exactly like this but replace "problem" with the name of the thing you want, like theorem or lemma or whatever
 
\begin{document}
 
%\renewcommand{\qedsymbol}{\filledbox}
%Good resources for looking up how to do stuff:
%Binary operators: http://www.access2science.com/latex/Binary.html
%General help: http://en.wikibooks.org/wiki/LaTeX/Mathematics
%Or just google stuff
 
\title{Infi1 Homework}
\author{KanHar}
\maketitle 
 
\begin{problem}{1}
	Prove: $\sqrt{3} \notin \mathbb{Q}$
\end{problem}

\begin{proof}{1}
	We will prove: $\forall n \in \mathbb{N}: n \nmid 3 \implies n^2 \nmid 3$
\begin{proof}{1.1} \label{p1_1}
	$$n \nmid 3 \implies n\bmod{3} \in \{1,2\}$$
	Option A: $$n \bmod{3} = 1 \newline$$
	$$n = (3m+1)$$
	$$n^2 = 9m^2 + 6m + 1$$
	$$9m^2\bmod3=0$$
	$$6m\bmod3=0$$
	$$n^2\bmod{3} = 0+0+1 = 1$$
	Option B: $$n \bmod{3} = 2 \newline$$
	$$n = (3m+2)$$
	$$n^2 = 9m^2 + 12m + 4$$
	$$9m^2\bmod3=0$$
	$$12m\bmod3=0$$
	$$4\bmod3=1$$
	$$n^2\bmod{3} = 0+0+1 = 1$$

	Therefore $n^2 \bmod{3} \ne 0 \implies  n^2 \nmid 3$.
\end{proof}
	Assume: $\sqrt{3} \in \mathbb{Q}$
	$$\exists{n,m}\in \mathbb{N}: \frac{n}{m} = \sqrt{3} \land \gcd(n,m)=1$$
	$$\frac{n^2}{m^2}=3$$
	$$n^2 = 3m^2$$
	By proof 1.1:
	$$n^2 \mid 3 \implies n \mid 3$$
	$$\exists k \in \mathbb{N}:n = 3k$$
	$$n^2 = 9k^2$$
	$$\frac{9k^2}{m^2}=3$$
	$$9k^2 = 3m^2$$
	$$m^2 = 3k^2$$
	$$m^2 \mid 3 \implies m \mid 3$$
	$$m \mid 3 \land n \mid 3 \implies \gcd(m,n) \ne 1$$
	We have reached a contradiction. Therefore our assumption is false and $\sqrt{3} \notin \mathbb{Q}$.
\end{proof}

\begin{problem}{2}
	Prove or falsify: $(x \ge 0 \in \mathbb{R}, \forall \epsilon > 0 \in \mathbb{R}: x<\epsilon) \implies x = 0$ \linebreak
\end{problem}

\begin{proof}{2}
	Assume $ x \ne 0 \implies x > 0$
	$$x \in R \implies \frac{x}{2} \in R$$
	$$x > 0 \implies x > \frac{x}{2}$$
	We have reached a contradiction, therefore, our assumption is wrong and $x=0$.
\end{proof}


\begin{problem}{3}
	Prove:
	$A \subseteq R, \exists \epsilon > 0: \forall a \in A: a > \epsilon \implies \lnot (0 = \inf A)$
\end{problem}

\begin{proof}{3}
	Assume $\inf A = 0$: \newline
	By the definition, $\epsilon$ is a lower bound of A. Therefore, 0 cannot be the infimum of the group as the infimum is defined as the highest lower bound, and $\epsilon > 0$ is a lower bound. \newline
	We have reached a contradiction, therefore the assumption is wrong and $\inf A \ne 0 \implies \lnot (0 = \inf A)$.
\end{proof}

\begin{problem}{4}
	$B = \{(-1)^{n-1}\cdot(2+\frac{3}{n})) | n \in \mathbb{N}\}$. Find upper and lower bounds, max and min.
\end{problem}

\begin{proof}{4}
	The function $\frac{1}{n}$ is monotonically decreasing, therefore $2+\frac{3}{n}$ is monotonically decreasing.
	There value of the expression $(-1)^{n-1}$ is dependent on $n \bmod{2} \in \{0,1\}$: \newline
	There are 2 possibilities - Option A - $$n \bmod{2} = 0$$
	$$\exists k \in \mathbb{N}: n = 2k$$
	$$(-1)^{n-1}\cdot(2+\frac{3}{n}) = (-1)^{2k-1}\cdot(2+\frac{3}{2k}) = (-1)^{2k}\cdot(-1)^{-1}\cdot(2+\frac{3}{2k})$$
	$$=-(2+\frac{3}{2k})$$
	This expression is monotonicaly increasing so it's infimum is at $k = 1$.
	$$\inf{-(2+\frac{3}{2k})}=-(2+\frac{3}{2})=-\frac{7}{2} \in B$$
	This expression is always negative. \newline
	Option B - $$n \bmod{2} = 1$$
	$$\exists k \in \mathbb{N}: n = 2k-1$$
	$$(-1)^{n-1}\cdot(2+\frac{3}{n}) = (-1)^{2k-2}\cdot(2+\frac{3}{2k-1}) = (2+\frac{3}{2k-1})$$
	This expression is monotonicaly decreasing so it's suprimum is at $k=1$.
	$$\sup{(2+\frac{3}{2k-1})} = (2+\frac{3}{2-1}) = 5 \in B$$
	This expression is always positive.
	As the first option is always negative and the second is always positive, the suprimum of the first option and the infimum of the second option hold for the whole expression. Both the suprimum and the infimum are inside the group so they are maximum and minimum
	Therefore, $$\max B = 5, \min B = -\frac{7}{2}$$
\end{proof}

\end{document}
              